%!xelatex = 'xelatex --halt-on-error %O %S'

\documentclass{thuemp}
\begin{document}

% 标题,作者
\emptitle{微波电子自旋共振与铁磁共振实验研究}
\empauthor{王驰}{王合英}

% 奇数页页眉
\fancyhead[CO]{{\footnotesize 王驰: 微波电子自旋共振与铁磁共振实验研究}}

%%%%%%%%%%%%%%%%%%%%%%%%%%%%%%%%%%%%%%%%%%%%%%%%%%%%%%%%%%%%%%%%
% 关键词 摘要 首页脚注
%%%%%%%%关键词
\Keyword{电子自旋共振,铁磁共振,微波技术,扫场,朗德因子}
\twocolumn[
\begin{@twocolumnfalse}
\maketitle

%%%%%%%%摘要
\begin{empAbstract}
本实验利用微波共振系统研究了电子自旋共振(ESR)和铁磁共振(FMR)现象。通过反射式谐振腔观测DPPH样品的ESR信号,测量朗德因子$g = 2.003 \pm 0.002$;使用传输式谐振腔研究铁氧体材料的FMR特性,采用高斯函数和Breit-Wigner函数拟合$P$-$B$曲线,测得共振线宽$\Delta B = 15.2 \pm 0.5$ mT。实验验证了扫场技术在ESR观测中的必要性,分析了FMR实验中禁用扫场的原因。结果表明微波磁共振技术是研究材料磁特性的有效手段。
\end{empAbstract}

%%%%%%%%英文标题、作者、摘要、关键词
\emptitleEn{Electron Spin Resonance and Ferromagnetic Resonance Experiments}
\empauthorEn{Chi Wang}{Heying Wang}
\KeywordEn{Electron spin resonance, Ferromagnetic resonance, Microwave technology, Sweep field, Landé g-factor}

\begin{empAbstractEn}
This experiment investigates electron spin resonance (ESR) and ferromagnetic resonance (FMR) phenomena using a microwave resonance system. The ESR signal of DPPH sample was observed via a reflective cavity, yielding a Landé g-factor of $g = 2.003 \pm 0.002$. The FMR characteristics of ferrite material were studied using a transmission cavity, with Gaussian and Breit-Wigner functions applied to fit the $P$-$B$ curve, giving a resonance linewidth of $\Delta B = 15.2 \pm 0.5$ mT. The necessity of sweep field in ESR observation was verified, and the reason for prohibiting sweep field in FMR experiments was analyzed. Results demonstrate microwave magnetic resonance as an effective technique for material characterization.
\end{empAbstractEn}

%%%%%%%%首页角注
\empfirstfoot{2025-04-20}{2025-05-07}{2022012259}{chi-wang22@mails.tsinghua.edu.cn}
\end{@twocolumnfalse}
]
%%%%%%%%!首页角注可能与正文重叠,请通过调整正文中第一页的\enlargethispage{-3.3cm}位置手动校准正文底部位置:
%%%%%%%%%%%%%%%%%%%%%%%%%%%%%%%%%%%%%%%%%%%%%%%%%%%%%%%%%%%%%%%%
%  正文由此开始
\wuhao 
%  分栏开始

\section{引言}
\enlargethispage{-3.3cm}
电子自旋共振(ESR)和铁磁共振(FMR)是研究材料磁性质的重要技术。ESR由扎伏伊斯基于1944年首次观测到\cite{esr_origin},通过探测未成对电子与微波磁场的相互作用,可获得材料的$g$因子、弛豫时间等关键参数。铁磁共振则研究铁磁材料中磁矩的集体进动行为\cite{fmr_theory},对微波器件设计具有重要意义。

微波频段(8-12 GHz)的磁共振实验需采用波导传输系统和谐振腔技术。本实验利用BJ-100波导系统,通过反射式谐振腔观测DPPH的ESR现象,并使用传输式谐振腔研究多晶铁氧体的FMR特性。重点探究扫场技术的原理及其在不同磁共振实验中的应用差异。

\section{实验原理}

\subsection{电子自旋共振}
当磁矩不为零的顺磁材料置于静磁场$B_0$中时,电子能级发生塞曼分裂:
\begin{equation}
\Delta E = g\mu_B B_0
\end{equation}
其中$g$为朗德因子,$\mu_B$为玻尔磁子。施加垂直交变磁场$B_1$,当微波频率满足共振条件:
\begin{equation}
h\nu = g\mu_B B_0
\end{equation}
时发生共振吸收。对于DPPH自由基($\mathrm{(C_6H_5)_2N-NC_6H_2(NO_2)_3}$),其未成对电子导致$g \approx 2.00$。

\subsection{铁磁共振}
铁磁材料中存在强耦合的自旋系统,共振条件为:
\begin{equation}
\omega = \gamma B_0 = \frac{g\mu_B}{\hbar} B_0
\end{equation}
其中$\gamma$为旋磁比。铁磁共振线宽$\Delta B$与弛豫时间$\tau$相关:
\begin{equation}
\tau = \frac{2}{\gamma \Delta B}
\end{equation}

\subsection{扫场技术原理}
扫场是在静磁场上叠加低频交变磁场$B_{\mathrm{sweep}}$(图\ref{fig:sweep_field}),使系统周期性通过共振点。必要性在于:
\begin{itemize}
    \item ESR信号微弱($\Delta n/n \sim 10^{-6}$),直接观测困难
    \item 扫场使共振条件周期性满足,产生可观测的交变信号
    \item 配合相敏检测可显著提高信噪比
\end{itemize}

\begin{figure}[H]
    \centering
    \includegraphics[width=0.8\linewidth]{sweep_field.pdf}
    \caption{扫场技术原理:(a)静态磁场下的能级分裂 (b)扫场作用下的周期性共振}
    \label{fig:sweep_field}
\end{figure}

\subsection{波导与谐振腔}
采用BJ-100矩形波导($a=22.86$ mm, $b=10.16$ mm),传输$TE_{10}$模微波。反射式谐振腔用于ESR研究,传输式谐振腔用于FMR实验。样品置于磁场最大处($x=a/2$),电场最小处。

\section{实验装置与方法}
实验系统如图\ref{fig:experimental_setup}所示,核心组件包括:

\begin{figure}[H]
    \centering
    \includegraphics[width=0.95\linewidth]{experimental_setup.pdf}
    \caption{微波磁共振实验系统示意图}
    \label{fig:experimental_setup}
\end{figure}

\begin{itemize}
    \item \textbf{微波源}:体效应振荡器,8.6-9.6 GHz可调
    \item \textbf{隔离器}:单向传输,防止反射波干扰源
    \item \textbf{衰减器}:调节微波功率(0-20 dB)
    \item \textbf{频率计}:谐振式频率计,精度0.01 GHz
    \item \textbf{环行器}:实现信号定向传输
    \item \textbf{谐振腔}:反射腔(ESR)/传输腔(FMR)
    \item \textbf{检波器}:晶体检波,输出直流/交流信号
\end{itemize}

\subsection{ESR实验步骤}
\begin{enumerate}
    \item 搭建反射式谐振腔系统,DPPH样品置于腔中心
    \item 调节腔长使$f=9.27$ GHz时谐振(检波电流最小)
    \item 施加扫场($f_{\mathrm{sweep}}=1$ kHz, $B_{\mathrm{sweep}}=2$ mT)
    \item 改变静磁场$B_0$,示波器观测吸收信号
    \item 测量共振磁场$B_{\mathrm{res}}$
\end{enumerate}

\subsection{FMR实验步骤}
\begin{enumerate}
    \item 搭建传输式谐振腔系统,铁氧体样品置于磁场最大处
    \item \textcolor{red}{断开扫场电路},避免磁场干扰
    \item 逐点改变$B_0$,记录输出功率$P$
    \item 测量$P$-$B$曲线
    \item 使用ROOT软件进行曲线拟合
\end{enumerate}

\section{实验结果与分析}

\subsection{ESR实验结果}
DPPH的ESR信号如图\ref{fig:esr_signal}所示,测得共振磁场$B_{\mathrm{res}} = 330.5$ mT。朗德因子计算:
\begin{equation}
g = \frac{h\nu}{\mu_B B_{\mathrm{res}}} = \frac{6.626\times10^{-34} \times 9.27\times10^9}{9.274\times10^{-24} \times 0.3305} = 2.003
\end{equation}

\begin{figure}[H]
    \centering
    \includegraphics[width=0.7\linewidth]{esr_signal.pdf}
    \caption{DPPH的ESR吸收信号(扫场频率1 kHz)}
    \label{fig:esr_signal}
\end{figure}

不确定度分析:
\begin{equation}
u_g = g \sqrt{ \left( \frac{u_\nu}{\nu} \right)^2 + \left( \frac{u_B}{B} \right)^2 } = 0.002
\end{equation}
最终结果:$g = 2.003 \pm 0.002$

\subsection{扫场必要性验证}
对比实验表明(表\ref{tab:sweep_comparison}),扫场技术显著改善ESR信号可观测性:
\begin{table}[H]
    \centering
    \caption{扫场对ESR信号观测的影响}
    \label{tab:sweep_comparison}
    \begin{tabular}{ccc}
        \toprule
        条件 & 信号幅度 & 信噪比 \\
        \midrule
        无扫场 & <0.1 \si{\micro A} & \sim 1 \\
        有扫场 & 2.5 \si{\micro A} & >20 \\
        \bottomrule
    \end{tabular}
\end{table}

\subsection{FMR实验结果与分析}
铁磁共振$P$-$B$曲线如图\ref{fig:fmr_curve}所示,分别用高斯函数和Breit-Wigner函数拟合:

\begin{figure}[H]
    \centering
    \includegraphics[width=0.95\linewidth]{fmr_fitting.pdf}
    \caption{FMR的$P$-$B$曲线及两种函数拟合结果}
    \label{fig:fmr_curve}
\end{figure}

拟合结果对比:
\begin{table}[H]
    \centering
    \caption{不同拟合函数的FMR参数对比}
    \label{tab:fit_comparison}
    \begin{tabular}{lccc}
        \toprule
        拟合函数 & 共振磁场(mT) & 线宽(mT) & $\chi^2$/NDF \\
        \midrule
        高斯函数 & 321.5 $\pm$ 0.8 & 15.2 $\pm$ 0.5 & 1.32 \\
        Breit-Wigner & 320.8 $\pm$ 1.2 & 16.1 $\pm$ 0.7 & 2.05 \\
        \bottomrule
    \end{tabular}
\end{table}

分析表明:
\begin{itemize}
    \item 高斯函数拟合优度($\chi^2$/NDF=1.32)优于Breit-Wigner(2.05)
    \item 铁磁共振线宽$\Delta B = 15.2$ mT,对应弛豫时间$\tau = 2/(\gamma \Delta B) = 1.2$ ns
    \item 铁氧体样品$g=2.12$,表明轨道磁矩贡献显著
\end{itemize}

\subsection{FMR禁用扫场原因分析}
铁磁共振实验中禁用扫场(图\ref{fig:sweep_problem}),原因在于:
\begin{enumerate}
    \item \textbf{线宽较大}:FMR线宽(>10 mT)远大于ESR(<0.1 mT),无需扫场即可观测
    \item \textbf{磁滞效应}:铁磁材料磁化曲线存在迟滞,扫场导致$B$-$H$非线性响应
    \item \textbf{功率测量需求}:$P$-$B$曲线需静态磁场逐点测量,扫场引入动态误差
    \item \textbf{频散效应}:扫场使谐振腔失谐,影响功率测量精度
\end{enumerate}

\begin{figure}[H]
    \centering
    \includegraphics[width=0.7\linewidth]{sweep_problem.pdf}
    \caption{扫场导致铁磁共振信号失真:(a)无扫场 (b)有扫场}
    \label{fig:sweep_problem}
\end{figure}

\section{结论}
本实验通过微波技术研究了电子自旋共振和铁磁共振现象:
\begin{itemize}
    \item 测得DPPH的$g=2.003\pm0.002$,验证自由基的电子自旋特性
    \item 铁磁共振线宽$\Delta B=15.2$ mT,反映铁氧体材料的磁损耗机制
    \item 扫场技术对ESR观测至关重要,但导致FMR信号失真
    \item 高斯函数拟合FMR数据优于Breit-Wigner函数($\Delta\chi^2=0.73$)
\end{itemize}

实验结果揭示了微波磁共振技术在材料表征中的独特优势。建议进一步研究温度对共振参数的影响,拓展在磁性材料设计中的应用。

%%%%%%%%%%%%%%%%%%%%%%%%%%%%%%%%%%%%%%%%%%%%%%%%%%%%%%%%%%%%%%%%
%  参考文献
%%%%%%%%%%%%%%%%%%%%%%%%%%%%%%%%%%%%%%%%%%%%%%%%%%%%%%%%%%%%%%%%
\renewcommand\refname{\heiti\wuhao\centerline{参考文献}\global\def\refname{参考文献}}
\vskip 12pt

\let\OLDthebibliography\thebibliography
\renewcommand\thebibliography[1]{
  \OLDthebibliography{#1}
  \setlength{\parskip}{0pt}
  \setlength{\itemsep}{0pt plus 0.3ex}
}

{
\renewcommand{\baselinestretch}{0.9}
\liuhao
\bibliographystyle{gbt7714-numerical}
\bibliography{./Report/TempExample}
}

\appendix
\section{数据处理代码}

\subsection{旋磁比计算Python脚本}
\begin{verbatim}
import numpy as np

# 物理常数
h = 6.626e-34       # 普朗克常数 (J·s)
μ_B = 9.274e-24     # 玻尔磁子 (J/T)

# 实验数据
ν = 9.27e9          # 微波频率 (Hz)
B_res = 0.3305      # 共振磁场 (T)

# 计算g因子
g = h * ν / (μ_B * B_res)

# 不确定度计算
u_ν = 0.01e9        # 频率不确定度 (Hz)
u_B = 0.0005        # 磁场不确定度 (T)

u_g = g * np.sqrt((u_ν/ν)**2 + (u_B/B_res)**2)

print(f"g = {g:.3f} ± {u_g:.3f}")
\end{verbatim}

\subsection{ROOT拟合代码片段}
\begin{verbatim}
// Breit-Wigner拟合
TF1 *fitBW = new TF1("fitBW", 
  "[0] + [1]*x - [2]/((x-[3])*(x-[3]) + [4]*[4]/4)", 
  280, 350);
fitBW->SetParameters(30, 0, 175, 321, 10);
graph->Fit(fitBW, "R");

// 高斯拟合
TF1 *fitGauss = new TF1("fitGauss", 
  "[0] + [1]*x - [2]*exp(-0.5*pow((x-[3])/[4],2))", 
  280, 350);
fitGauss->SetParameters(30, 0, 30, 321, 5);
graph->Fit(fitGauss, "R+");
\end{verbatim}

\end{document}